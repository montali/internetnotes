\documentclass[11pt]{article}
\usepackage[italian]{babel}
\usepackage[utf8]{inputenc}
\usepackage{graphicx}
\usepackage{float}
\usepackage{amsmath}
\usepackage{amsfonts}
\usepackage{hyperref}
\usepackage{glossaries}
\usepackage{dirtytalk}

\usepackage[normalem]{ulem}
\newcommand{\code}[1]{\texttt{#1}}
\newcommand{\numpy}{{\tt numpy}}    % tt font for numpy
\topmargin -.5in
\textheight 9in
\oddsidemargin -.25in
\evensidemargin -.25in
\textwidth 7in
\begin{document}

% ========== Edit your name here
\author{Simone Montali\\monta.li}
\title{Riassunti di Crittografia - Tecnologie Internet}

\maketitle

\medskip
\section{Concetti generali}
Prima di tutto, definiamo il significato di \textbf{computer security}: la protezione applicata ad un sistema informativo con lo scopo di ottenere integrità, disponibilità e confidenzialità delle risorse. Emerge un concetto molto importante: la \textbf{CIA triad}, ossia confidenzialità, integrità, availability. (ricordiamo però altri due obiettivi: autenticità e accountability).
\subsection{CIA Triad}
\subsubsection{Confidentiality}
La confidentiality ha l'obiettivo di preservare restrizioni sull'accesso alle informazioni, inclusa la privacy personale ed informazioni proprietarie. Con \textbf{data confidentiality} intendiamo che informazioni confidenziali non sono rese visibili ad individui non autorizzati. Con \textbf{privacy} intendiamo che ogni individuo decide quali informazioni che lo riguardano rendere disponibili, e a chi. 
\subsubsection{Integrity}
L'integrity protegge dalla modifica o distruzione di informazioni, includendo non-repudiation e authenticity. Una perdita di integrity è la modifica non autorizzata di informazioni. Con \textbf{data integrity} intendiamo l'assicurarsi che informazioni e programmi vengano cambiati in maniera definita. Con \textbf{system integrity} intendiamo l'assicurarsi che un sistema svolga le sue funzioni in maniera corretta, libero da manipolazioni. 
\subsubsection{Availability}
Con availability intendiamo l'accesso affidabile alle informazioni. Una perdita di availability è l'interruzione dell'accesso ad alcune risorse.
\subsubsection{Authenticity}
Con authenticity intendiamo la proprietà, delle informazioni, di essere genuine e verificabili. In pratica, la verifica che gli utenti siano chi dicono di essere.

\subsubsection{Accountability}
Il goal dell'accountability è quello di poter tracciare tutte le azioni di un'entità sul sistema, in modo da riconoscere i colpevoli di un eventuale security breach. 

\subsection{Sfide della computer security}
Elenchiamo ora alcune sfide a cui la computer security deve sopperire:
\begin{enumerate}
    \item I requirements di sicurezza sembrano semplici, ma i meccanismi per risolverli sono complessi
    \item Nello sviluppo di un meccanismo/algoritmo di sicurezza, bisogna sempre considerare i potenziali attacchi
    \item Per il punto precedente, spesso le procedure necessarie sono controintuitive
    \item Dopo aver progettato i sistemi di sicurezza, bisogna decidere dove utilizzarli
    \item I meccanismi di sicurezza coinvolgono spesso più di un algoritmo/protocollo
    \item Il vantaggio per un malintenzionato è chiaro: a lui basta trovare una sola falla, mentre il progettista deve coprirle tutte
    \item C'è una naturale tendenza da parte di utenti/manager a non notare i benefici della sicurezza finché è troppo tardi 
    \item La sicurezza richiede monitoring costante e regolare
    \item La sicurezza è spesso un aggiunta successiva alla progettazione, piuttosto che parte integrante
    \item Molti utenti/amministratori vedono la sicurezza come un impedimento alle operazioni
\end{enumerate}
\subsection{OSI security architecture}
Definiamo alcuni termini:
\paragraph{Security attack} Ogni azione che compromette la sicurezza delle informazioni possedute da un'organizzazione
\paragraph{Security mechanism} Un processo progettato per rilevare, prevenire e recuperare attacchi di sicurezza
\paragraph{Security service} Un servizio che migliora la sicurezza del data processing/transfer di un'organizzazione. Sono progettati come antagonisti degli attacchi di sicurezza, e fanno utilizzo di meccanismi di sicurezza
\paragraph{Threat} Il potenziale per una violazione di sicurezza, che esiste quando c'è una circostanza, possibilità, azione o evento che potrebbe mettere a rischio la sicurezza.
\paragraph{Attack} Un assalto alla sicurezza di sistema che deriva da un intelligent threat, ossia un tentativo deliberato di evadere i sistemi di sicurezza.
\subsection{Security attacks}
Gli \textbf{attacchi attivi} coinvolgono qualche modifica del data stream, mentre quelli passivi sono di 4 tipi:
\begin{itemize}
    \item \textbf{spoofing}: attacca l'authenticity
    \item \textbf{tampering}: attacca l'integrity 
    \item \textbf{replay/reflection}: attacca l'authenticity
    \item \textbf{Denial Of Service}: attacca l'availability
\end{itemize}
È difficile prevenire gli attacchi attivi perché il numero di vulnerabilità è troppo alto: il goal è minimizzarne i danni. 
\subsection{Security service}
Un \textbf{security service} è un servizio di comunicazione/processing fornito da un sistema per dare specifici tipi di protezione a risorse; implementa security policies ed è implementato da security mechanisms. Fornisce diverse tipologie di sicurezza.
\subsubsection{Confidentiality}
Protezione verso accesso ai dati non autorizzato. È collegato a dati ed anonimità.
\subsubsection{Data integrity e message authentication}
La data integrity è la proprietà che i dati non siano stati cambiati, distrutti o persi. Protegge contro modifiche non autorizzate, rilevando cambiamenti. La data origin authentication certifica la fonte di un dato, verificandone l'identità. La message authentication è l'insieme delle due cose.
\subsubsection{Peer entity authentication}
Fornisce la conferma dell'identità di un peer in un'associazione. Due entità sono considerate peers se implementano lo stesso protocollo in sistemi diversi. L'authentication è utilizzata nello stabilimento della connessione o durante il trasferimento. Prova a fornire anche l'assicurazione che un'entità non sia mascherata o stia replicando una connessione passata.

\subsubsection{Authorization e access control}
L'authorization è la verifica dei permessi su una risorsa/sistema. L'access control è l'abilità di limitare e controllare l'accesso ad un sistema. 
\subsubsection{System integrity and availability}
La system integrity è la qualità che un sistema ha quando può eseguire la sua funzione. Si ottiene proteggendo il sistema da modifiche, perdite, distruzione. L'availability è la proprietà di un sistema di essere accessibile ed utilizzabile quando necessario, concordando con le specifiche della performance del sistema. 
\subsubsection{Accountability e non-repudiation}
L'accountability è la proprietà di un sistema/risorsa che assicura che le azioni di un'entità siano tracciabili a quell'entità. L'\textbf{audit} è un sistema che salva informazioni necessarie all'accountability.
La non-repudiation fornisce protezione verso il falso rinnego di azioni. 
\subsection{Meccanismi di sicurezza}
I meccanismi di sicurezza hanno relazioni coi servizi; citiamo:
\begin{itemize}
    \item Cifratura
    \item Firma digitale
    \item Access control
    \item Verifica dell'integrità dei dati
    \item Scambio di autenticazione
    \item Traffic padding
    \item Routing control
    \item Notarization
\end{itemize}
\subsection{Principi di security design}
Elenchiamo alcuni principi di security design:
\paragraph{Economy of mechanism} Significa che il design di misure di sicurezza dovrebbe essere il più semplice possibile.
\paragraph{Fail-safe defaults} Il concetto è basarsi sui permessi, piuttosto che l'esclusione. La situazione di default è mancanza di accesso.
\paragraph{Complete mediation} Significa che ogni accesso deve essere verificato tramite il meccanismo di access control.
\paragraph{Open design} Mentre le encryption keys devono essere segrete, gli algoritmi devono essere pubblici.
\paragraph{Separazione di privilegi} Attributi di privilegio multipli sono necessari per l'accesso ad una risorsa restricted.
\paragraph{Least privilege} Ogni processo deve operare con il numero più basso di permessi possibile.
\paragraph{Least common mechanism} Il design deve minimizzare le funzioni utilizzate da più utenti, fornendo sicurezza mutual.
\paragraph{Psychological acceptability} I meccanismi di sicurezza non devono interferire con il lavoro degli utenti. (least astonishment)
\paragraph{Isolation} È un principio che si applica a tre contesti: 
\begin{enumerate}
    \item Sistemi di accesso pubblici, che devono essere isolati da risorse critiche
    \item Processi e file di utenti individuali devono essere isolati gli uni gli altri 
    \item I meccanismi di security devono essere isolati: non dev'essere possibile accedervi
\end{enumerate} 
\paragraph{Modularity} Si riferisce al separamento delle funzioni di sicurezza in moduli, ed all'architettura modulare per il design.
\paragraph{Layering} Si riferisce all'utilizzo di approcci multipli di protezione indirizzati a persone, tecnologia, operazioni.
\subsection{Attack surface e trees}
Un'\textbf{attack surface} consiste nelle vulnerabilità raggiungibili di un sistema. Un \textbf{attack tree} è una struttura gerarchica ad albero che rappresenta le tecniche di exploit delle vulnerabilità. 
\subsection{Modello per la network security}
Tutte le tecniche di sicurezza hanno due componenti: una trasformazione sulle informazioni, e un segreto condiviso. Consideriamo 4 tasks semplici per il design di un servizio:
\begin{enumerate}
    \item Progettare un algoritmo per la trasformazione 
    \item Generare l'informazione segreta
    \item Sviluppare metodi per la distribuzione e condivisione del segreto
    \item Specificare un protocollo utilizzabile dalle entità che partecipano all'algoritmo
\end{enumerate}





\end{document}
